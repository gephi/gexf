\documentclass[a4paper,10pt]{article}

\usepackage[utf8x]{inputenc}
\usepackage{makeidx}
\usepackage{listings}
\usepackage{color}
\usepackage{graphicx}
\usepackage{hyperref}

%\usepackage{html}   %  *always* load this for LaTeX2HTML
%\begin{htmlonly}
%  \usepackage{verbatim}
%  \providecommand{\lstlisting}[2][]{\verbatiminput{#2}}
%\end{htmlonly}

\input colordvi
\definecolor{lightblue}{rgb}{.3,.5,1}
\definecolor{orange}{rgb}{1,.7,0}
\definecolor{darkorange}{rgb}{1,.4,0}
\definecolor{darkgreen}{rgb}{0,.4,0}
\definecolor{darkblue}{rgb}{0,0,.4}
\definecolor{darkred}{rgb}{.56,0,0}
\definecolor{gray}{rgb}{.3,.3,.3}
\definecolor{darkgray}{rgb}{.2,.2,.2}
\definecolor{shadecolor}{gray}{0.925}

\lstdefinelanguage{GEXF}{
    morekeywords=[0]{gexf, meta, graph, creator, description, nodes, edges, node, edge, attributes, attribute, options, attvalues, attvalue, default, keywords, color, size, position, parents, parent},
    morekeywords=[1]{id, for, pid, label, source, target, mode, defaultedgetype, type, lastmodifieddate, datefrom, dateto, cardinal, class, count, title, value, r, g, b, x, y, z},
    morekeywords=[2]{simple, double, static, dynamic, integer, double, float, boolean, liststring, string},
    morestring=[b]",
}

\lstdefinestyle{gexf}{
    language=GEXF,
    xleftmargin=\parindent,
    xrightmargin=\parindent,
    aboveskip=3mm,
    belowskip=3mm,
    tabsize=2,
    columns=[l]fullflexible,
    showstringspaces=false,
    % text styles
    basicstyle=\scriptsize\ttfamily,
    commentstyle=\footnotesize\rmfamily\em,
    stringstyle=\rmfamily\em\color{gray},
    keywordstyle=[0]\color{blue},
    keywordstyle=[1]\color{darkorange},
    keywordstyle=[2]\underbar,
    % numbers
    numbers=left,
    numberstyle=\tiny,
    stepnumber=3,
    firstnumber=1,
    % decoration
    frame=shadowbox,
    frameround=tttf
    %backgroundcolor=\color{shadecolor}
}

\lstdefinelanguage{RNC}{
    morekeywords=[1]{start,element,attribute,text,empty,string,text},
    morekeywords=[2]{default,namespace,datatypes},
    morekeywords=[3]{grammar,include,parent,inherit},
    morekeywords=[4]{xsd},
    morekeywords=[5]{a,defaultValue},
    morestring=[b]",
    morecomment=[l]{\#},
    alsodigit={-},
    sensitive
}


\lstdefinestyle{rnc}{
    language=RNC,
    basicstyle=\scriptsize\ttfamily,
    commentstyle=\footnotesize\sffamily\em\color{darkgreen},
    stringstyle=\sffamily\color{red},
    keywordstyle=[1]\color{blue}\bfseries,
    keywordstyle=[2]\color{lightblue}\bfseries,
    keywordstyle=[3]\color{lightblue}\bfseries,
    keywordstyle=[4]\color{darkgreen}\bfseries,
    keywordstyle=[5]\color{orange}\bfseries
}

%\lstdefinestyle{xml}{
    language=XML,
    xleftmargin=\parindent,
    xrightmargin=\parindent,
    aboveskip=3mm,
    belowskip=3mm,
    basicstyle=\footnotesize\ttfamily,
    commentstyle=\small\rmfamily\em,
    tabsize=2,
    columns=[l]fullflexible,
    showstringspaces=false,
    % keywords
    keywordstyle=\color{blue}\bfseries,
    stringstyle=\color{red},
    usekeywordsintag=false,
    markfirstintag=true,
    % numbers
    numbers=left,
    numberstyle=\tiny,
    stepnumber=3,
    firstnumber=1,
    % decoration
    frame=shadowbox,
    frameround=tttf,
    backgroundcolor=\color{shadecolor}
}


\hypersetup{
    bookmarks=true,         % show bookmarks bar?
    unicode=false,          % non-Latin characters in Acrobat’s bookmarks
    pdftoolbar=true,        % show Acrobat’s toolbar?
    pdfmenubar=true,        % show Acrobat’s menu?
    pdffitwindow=false,     % window fit to page when opened
    pdfstartview={FitH},    % fits the width of the page to the window
    pdftitle={GEXF 1.2draft Primer},    % title
    pdfauthor={Sebastien Heymann},     % author
    pdfsubject={GEXF 1.2draft Primer},   % subject of the document
    pdfcreator={Gephi},   % creator of the document
    pdfproducer={Gephi}, % producer of the document
    pdfkeywords={gephi gexf graph}, % list of keywords
    pdfnewwindow=true,      % links in new window
    colorlinks=false,       % false: boxed links; true: colored links
    linkcolor=red,          % color of internal links
    citecolor=green,        % color of links to bibliography
    filecolor=magenta,      % color of file links
    urlcolor=cyan           % color of external links
}

\makeindex


%opening
\title{GEXF 1.3 Primer}
\author{GEXF Working Group}

\begin{document}

\maketitle

\begin{abstract}
GEXF Primer is a non-normative document intended to provide an easily readable description of the GEXF facilities, and is oriented towards quickly understanding how to create GEXF documents. This primer describes the language features through examples which are complemented by references to normative texts. Specification is in \href{http://relaxng.org/compact-tutorial-20030326.html}{RelaxNG Compact} grammar.
\end{abstract}

\tableofcontents

\section{Introduction} \label{introduction}

\paragraph{}
This document, GEXF Primer, provides an description of GEXF, and should be used alongside the formal descriptions of the language contained in the GEXF specification. The intended audience of this document includes application developers whose programs read and write GEXF files, and users who want to communicate with programs using GEXF import/export. The text assumes that you have a basic understanding of XML 1.0 and  XML-Namespaces. Basic knowledge of XML Schema is also assumed for some parts of this document. Each major section of the primer introduces new features of the language, and describes those features in the context of concrete examples.

\paragraph{}
Section 2 covers the basic mechanisms of GEXF. It describes how to declare a simple graph by defining its nodes and edges and how to add simple user data to the graph.

\paragraph{}
Section 5 describes dynamic graph model (i.e. graphs over time).

\paragraph{}
Section 6 describes mechanisms for extending GEXF to add specific data with the Visualization module in example.

\paragraph{}
The primer is a non-normative document, which means that it does not provide a definitive specification of the GEXF language. The examples and other explanatory material in this document are provided to help you understand GEXF, but they may not always provide definitive answers. In such cases, you will need to refer to the GEXF specification, and to help you do this, we provide many links pointing to the relevant parts of the specification.

\paragraph{}
In this document, we may use the terms network and graph interchangeably.

\section{Basic Concepts} \label{basic}

The purpose of a GEXF document is to define a graph representing a network. Let us start by considering the minimal graph shown in the figure below. It contains 2 nodes and 1 edge.

\begin{figure}[!ht]
  \begin{center}
  \includegraphics[scale=0.15]{res/simple.png}
  \caption{Hello-world graph}
  \end{center}
\end{figure}

\subsection{A Simple Graph}

This is a dummy graph:

\lstset{ style=gexf }
\begin{lstlisting}[caption={Hello world!},label=helloworld]
<?xml version="1.0" encoding="UTF-8"?>
<gexf xmlns="http://www.gexf.net/1.3"
       xmlns:xsi="http://www.w3.org/2001/XMLSchema-instance"
       xsi:schemaLocation="http://www.gexf.net/1.3
                             http://www.gexf.net/1.3/gexf.xsd"
      version="1.3">
  <meta lastmodifieddate="2009-03-20">
    <creator>Gephi.org</creator>
    <description>A hello world! file</description>
  </meta>
  <graph defaultedgetype="directed">
    <nodes>
      <node id="0" label="Hello"/>
      <node id="1" label="Word"/>
    </nodes>
    <edges>
      <edge id="0" source="0" target="1"/>
    </edges>
  </graph>
</gexf>
\end{lstlisting}

The GEXF document consists of a gexf element and a variety of subelements: graph, node, edge. In the remainder of this section we will discuss these elements in detail and show how they define a graph.

\subsection{Header}

In this section we discuss the parts of the document which are common to all GEXF documents, basically the gexf element and the meta declaration.

\lstset{ style=gexf }
\begin{lstlisting}[caption={Header},label=header]
<?xml version="1.0" encoding="UTF-8"?>
<gexf xmlns="http://www.gexf.net/1.3"
       xmlns:xsi="http://www.w3.org/2001/XMLSchema-instance"
       xsi:schemaLocation="http://www.gexf.net/1.3
                             http://www.gexf.net/1.3/gexf.xsd"
      version="1.3">
  <meta lastmodifieddate="2009-03-20">
    <creator>Gephi.org</creator>
    <description>A hello world! file</description>
    <keywords>basic, web</keywords>
  </meta>
  ...
</gexf>
\end{lstlisting}

\paragraph{}
The first line of the document is an XML process instruction which defines that the document adheres to the XML 1.0 standard and that the encoding of the document is UTF-8, the standard encoding for XML documents. Of course other encodings can be chosen for GEXF documents.

\paragraph{}
The second line contains the root-element element of a GEXF document: the gexf element. The gexf element, like all other GEXF elements, belongs to the namespace \begin{footnotesize}http://www.gexf.net/1.3\end{footnotesize}. For this reason we define this namespace as the default namespace in the document by adding the XML Attribute \begin{footnotesize}xmlns="http://www.gexf.net/1.3"\end{footnotesize} to it. The two other XML Attributes are needed to specify the XML Schema for this document. In our example we use the standard schema for GEXF documents located on the gexf.net server. The first attribute, \begin{footnotesize}xmlns:xsi="http://www.w3.org/2001/XMLSchema-instance"\end{footnotesize}, defines xsi as the XML Schema namespace. The second attribute, \begin{footnotesize}xsi:schemaLocation ="http://www.gexf.net/1.3 http://www.gexf.net/1.3/gexf.xsd"\end{footnotesize}, defines the XML Schema location for all elements in the GEXF namespace.

\paragraph{}
The XML Schema reference is not required but it provides means to validate the document and is therefore strongly recommended.

\paragraph{}
The \begin{footnotesize}meta\end{footnotesize} element contains additionnal information about the network. Element leafs are assumed to be text, and \begin{footnotesize}lastmodifieddate\end{footnotesize} is an international standard date (yyyy-mm-dd). The \begin{footnotesize}graph\end{footnotesize} element must be declared after the \begin{footnotesize}meta\end{footnotesize} element.

\paragraph{}
GEXF document is specified in the \href{http://relaxng.org/compact-tutorial-20030326.html}{RelaxNG Compact} file \href{http://www.gexf.net/1.3/gexf.rnc}{gexf.rnc}. Header is ruled by the following declaration :

\lstset{ style=rnc }
\begin{lstlisting}[caption={Header Specification},label=headerRNC]
default namespace = "http://www.gexf.net/1.3"
namespace     rng = "http://relaxng.org/ns/structure/1.0"
datatypes     xsd = "http://www.w3.org/2001/XMLSchema-datatypes"

# Grammar root
start = element gexf { gexf-content }

# Tree
gexf-content =
    attribute version { string "1.3" }
  & attribute variant { xsd:string }?
  & (element meta { meta-content }?
  ,  element graph { graph-content })

# Attributes & Leafs
meta-content =
    attribute lastmodifieddate { xsd:date }?
  & element creator { text }?
  & element keywords { text }?
  & element description { text }?
\end{lstlisting}

\subsection{Network Topology} \label{networktopology}

The network topology structure containing nodes and edges is called the graph. A graph is, not surprisingly, denoted by a \begin{footnotesize}graph\end{footnotesize} element. Nested inside a graph element are the declarations of nodes and edges. A node is declared with the \begin{footnotesize}node\end{footnotesize} element inside a \begin{footnotesize}nodes\end{footnotesize} element, and an egde with the \begin{footnotesize}edge\end{footnotesize} element inside an \begin{footnotesize}edges\end{footnotesize} element. Nodes and edges order doesn't matter.

\lstset{ style=gexf }
\begin{lstlisting}[caption={The definition of the graph},label=topology]
<graph defaultedgetype="directed">
  <nodes>
    <node id="0" label="Hello" />
    <node id="1" label="Word" />
    ...
  </nodes>
  <edges>
    <edge id="0" source="0" target="1" weight="3.167" />
    ...
  </edges>
</graph>
\end{lstlisting}

\subsubsection{Declaring a Graph}

Graphs in GEXF are mixed, in other words, they can contain directed and undirected edges at the same time. If no direction is specified when an edge is declared, the default direction \begin{footnotesize}defaultedgetype\end{footnotesize} is applied to the edge. If you know what kind of edges are stored, you may interpret the mixed graph as a directed or an undirected graph at your own risks.

The default direction is declared as the optional XML-attribute \begin{footnotesize}defaultedgetype\end{footnotesize} of the \begin{footnotesize}graph\end{footnotesize} element. The three possible values for this XML-attribute are \textit{directed}, \textit{undirected} and \textit{mutual}. Note that the default direction is optional and would be assumed as \textit{undirected}.

The optional XML-attribute \begin{footnotesize}mode\end{footnotesize} set the kind of network: static or dynamic. Last one provides time support (see  the section \ref{dynamics} on Dynamics). Static mode is assumed by default.

The \begin{footnotesize}edges\end{footnotesize} element must be declared after the \begin{footnotesize}nodes\end{footnotesize} element.

\lstset{ style=gexf }
\begin{lstlisting}[caption={An empty graph!},label=aGraph]
<graph>
  <nodes>
  </nodes>
  <edges>
  </edges>
</graph>
\end{lstlisting}

\lstset{ style=rnc }
\begin{lstlisting}[caption={Topology Specification},label=topologyRNC]
graph-content =
    attribute defaultedgetype { defaultedgetype-type }?
  & attribute idtype { idtype-type }?
  & attribute mode { mode-type }?
  & (element nodes { nodes-content }
  ,  element edges { edges-content })


# Nodes

nodes-content =
    attribute count { xsd:nonNegativeInteger }?
  & element node { node-content }*

node-content =
    attribute id { id-type }
  & attribute label { xsd:token }?


# Edges

edges-content =
    attribute count { xsd:nonNegativeInteger }?
  & element edge { edge-content }*

edge-content =
    attribute id { id-type }
  & attribute type { edgetype-type }?
  & attribute label { xsd:token }?
  & attribute source { id-type }
  & attribute target { id-type }
  & attribute weight { weight-type }?


# Datatypes

defaultedgetype-type = [ a:defaultValue = "undirected" ]
    string "directed" |
    string "undirected" |
    string "mutual"

edgetype-type = [ a:defaultValue = "undirected" ]
    string "directed" |
    string "undirected" |
    string "mutual"

id-type =
    xsd:string | xsd:integer

idtype-type = [ a:defaultValue = "string" ]
    string "integer" |
    string "string"

mode-type = [ a:defaultValue = "static" ]
    string "static" |
    string "dynamic"

weight-type = [ a:defaultValue = "1.0" ]
    xsd:float
\end{lstlisting}

\subsubsection{Declaring a Node}

Nodes in the graph are declared by the \begin{footnotesize}node\end{footnotesize} element. Each node has an identifier, which must be unique within the entire document, i.e., in a document there must be no two nodes with the same identifier. The identifier of a node is defined by the XML-attribute \begin{footnotesize}id\end{footnotesize}, which is a string. Each node also may have a XML-attribute \begin{footnotesize}label\end{footnotesize} that acts as a description, which is a string.

\lstset{ style=gexf }
\begin{lstlisting}[caption={A node!},label=aNode]
<node id="0" label="Hello world" />
\end{lstlisting}

\lstset{ style=rnc }
\begin{lstlisting}[caption={Node Specification},label=nodeRNC]
node-content =
    attribute id { id-type }
  & attribute label { xsd:token }?

id-type =
    xsd:string | xsd:integer
\end{lstlisting}

\subsubsection{Declaring an Edge}

Edges in the graph are declared by the \begin{footnotesize}edge\end{footnotesize} element. Each edge must define its two endpoints with the XML-Attributes \begin{footnotesize}source\end{footnotesize} and \begin{footnotesize}target\end{footnotesize}. The value of the source, resp. target, must be the identifier of a node in the same document. The identifier of an edge is defined by the XML-Attribute \begin{footnotesize}id\end{footnotesize}. There is no order notion applied to edges.

\paragraph{}
Edges with only one endpoint, also called loops, selfloops, or reflexive edges, are defined by having the same value for source and target.

\paragraph{}
Each edge can have a optional XML-attribute \begin{footnotesize}label\end{footnotesize}, which is a string.

\paragraph{}
The optional XML-attribute \begin{footnotesize}type\end{footnotesize} declares if the edge is \textit{directed}, \textit{undirected} or \textit{mutual} (directed \textit{from source to target and from target to source}). If the direction is not explicitely defined, the default direction is applied to this edge as defined in the enclosing graph via the `defaultedgetype` attribute. The edge's endpoint are called source and target regardless whether the edge is directed or not.

\paragraph{}
The weight of the edge is set by the optional XML-attribute \begin{footnotesize}weight\end{footnotesize} and is a double. By default, the weight is \begin{footnotesize}1.0\end{footnotesize} and zero should be avoided although it's not explicitly forbidden.

\paragraph{}
Assuming two nodes having respectively the \begin{footnotesize}id\end{footnotesize} value set to \textit{0} and \textit{1}:

\lstset{ style=gexf }
\begin{lstlisting}[caption={An edge!},label=anEdge]
<edge id="0" source="0" target="1"/>
\end{lstlisting}

\lstset{ style=gexf }
\begin{lstlisting}[caption={A more complete edge},label=aMoreEdge]
<edge id="0" source="0" target="1" type="directed" weight="2.4" />
\end{lstlisting}

\lstset{ style=rnc }
\begin{lstlisting}[caption={Edge Specification},label=edgeRNC]
edge-content =
    attribute id { id-type }
  & attribute type { edgetype-type }?
  & attribute label { xsd:token }?
  & attribute source { id-type }
  & attribute target { id-type }
  & attribute weight { weight-type }?

# Datatypes

id-type =
    xsd:string | xsd:integer

edgetype-type = [ a:defaultValue = "undirected" ]
    string "directed" |
    string "undirected" |
    string "mutual"

weight-type = [ a:defaultValue = "1.0" ]
    xsd:float
\end{lstlisting}

\subsubsection{Parallel edges}

A multigraph is a graph where multiple edges can exist between two nodes.
GEXF supports this type of graph.
Note that this is different from hypergraph, which GEXF doesn't support.

\paragraph{}
Parallel edges must include the additional \textit{kind} attribute to characterise the edge. The triplet source-target-kind must still be unique. In other words, only parallel edges of different kinds can exist in GEXF. The \textit{kind} attribute is a string. In case the graph doesn't have any parallel edges, \textit{kind} can be omitted.

\lstset{ style=gexf }
\begin{lstlisting}[caption={Parallel edges},label=paralelEdge]
<edge id="0" source="0" target="1" weight="1.0" kind="friend"/>
<edge id="1" source="0" target="1" weight="1.0" kind="neighbor"/>
\end{lstlisting}

\subsection{Network Data} \label{networkdata}

\paragraph{}
In the previous section we discussed how to describe the topology of a graph in GEXF. While pure topological information may be sufficient for some applications, these days focus is made on network analysis based on data attributes.

\paragraph{}
A bunch of data can be stored within attributes. The concept is the same as table data or SQL. An attribute has a title/name and a value. Attribute’s name/title must be declared for the whole graph. It could be for instance “degree”, “valid” or “url”. Besides the name of the attribute a column also contains the type.

\subsubsection{Data types}

\paragraph{}
GEXF uses the XML Schema Data Types (\href{http://www.w3.org/TR/xmlschema-2/}{XSD 1.1}) for the following primitives: \href{http://www.w3.org/TR/xmlschema-2/#string}{string}, \href{http://www.w3.org/TR/xmlschema-2/#integer}{integer}, \href{http://www.w3.org/TR/xmlschema-2/#long}{long}, \href{http://www.w3.org/TR/xmlschema-2/#float}{float}, \href{http://www.w3.org/TR/xmlschema-2/#double}{double}, \href{http://www.w3.org/TR/xmlschema-2/#boolean}{boolean}, \href{http://www.w3.org/TR/xmlschema-2/#short}{short}, \href{http://www.w3.org/TR/xmlschema-2/#byte}{byte}, \href{http://www.w3.org/TR/xmlschema-2/#date}{date}, and \href{http://www.w3.org/TR/xmlschema-2/#anyURI}{anyURI}.

\paragraph{}
In addition, GEXF supports additional Data Types: bigdecimal, biginteger, char, liststring, listboolean, listinteger, listlong, listfloat, listdouble, listbyte, listshort, listbigdecimal, listinteger and listchar.

\subsubsection{Attributes Example}

Each Node of this graph has three attributes : an url, an indegree value and a boolean for french websites which is set to \textit{true} by default.

\lstset{ style=gexf }
\begin{lstlisting}[caption={A (small) Web Graph},label=webgraph]
<?xml version="1.0" encoding="UTF-8"?>
<gexf xmlns="http://www.gexf.net/1.3"
       xmlns:xsi="http://www.w3.org/2001/XMLSchema-instance"
       xsi:schemaLocation="http://www.gexf.net/1.3
                             http://www.gexf.net/1.3/gexf.xsd"
      version="1.3">
  <meta lastmodifieddate="2009-03-20">
    <creator>Gephi.org</creator>
    <description>A Web network</description>
  </meta>
  <graph defaultedgetype="directed">
    <attributes class="node">
      <attribute id="0" title="url" type="string"/>
      <attribute id="1" title="indegree" type="float"/>
      <attribute id="2" title="frog" type="boolean">
        <default>true</default>
      </attribute>
    </attributes>
    <nodes>
      <node id="0" label="Gephi">
        <attvalues>
          <attvalue for="0" value="http://gephi.org"/>
          <attvalue for="1" value="1"/>
        </attvalues>
      </node>
      <node id="1" label="Webatlas">
        <attvalues>
          <attvalue for="0" value="http://webatlas.fr"/>
          <attvalue for="1" value="2"/>
        </attvalues>
      </node>
      <node id="2" label="RTGI">
        <attvalues>
          <attvalue for="0" value="http://rtgi.fr"/>
          <attvalue for="1" value="1"/>
        </attvalues>
      </node>
      <node id="3" label="BarabasiLab">
        <attvalues>
          <attvalue for="0" value="http://barabasilab.com"/>
          <attvalue for="1" value="1"/>
          <attvalue for="2" value="false"/>
        </attvalues>
      </node>
    </nodes>
    <edges>
      <edge id="0" source="0" target="1"/>
      <edge id="1" source="0" target="2"/>
      <edge id="2" source="1" target="0"/>
      <edge id="3" source="2" target="1"/>
      <edge id="4" source="0" target="3"/>
    </edges>
  </graph>
</gexf>
\end{lstlisting}


\subsubsection{Declaring Attributes}

Attributes are declared inside an \begin{footnotesize}attributes\end{footnotesize} element. The XML-attribute \begin{footnotesize}class\end{footnotesize} apply nested attributes on nodes (\textit{node} value) or edges (\textit{edge} value). You need to specify the data type in \begin{footnotesize}type\end{footnotesize} and may specify a default value.

\lstset{ style=gexf }
\begin{lstlisting}[caption={Attributes Definition},label=attributesDef]
<graph mode="static">
  <attributes class="node">
    <attribute id="0" title="my-text-attribute" type="string"/>
    <attribute id="1" title="my-int-attribute" type="integer"/>
    <attribute id="2" title="my-bool-attribute" type="boolean"/>
  </attributes>
  <attributes class="edge">
    <attribute id="0" title="my-float-attribute" type="float">
      <default>2.0</default>
    </attribute>
  </attributes>
  ...
</graph>
\end{lstlisting}

\lstset{ style=rnc }
\begin{lstlisting}[caption={Attributes Specification},label=attributesRNC]
attributes-content =
    attribute class { class-type }
  & attribute mode { mode-type }?
  & element attribute { attribute-content }*

attribute-content =
    attribute id { id-type }
  & attribute title { xsd:string }
  & attribute type { attrtype-type }
  & element default { text }?
  & element options { text }?


# Datatypes

class-type =
    string "node" |
    string "edge"

mode-type = [ a:defaultValue = "static" ]
    string "static" |
    string "dynamic"

attrtype-type =
    string "integer" |
    string "long" |
    string "double" |
    string "float" |
    string "boolean" |
    string "liststring" |
    string "string" |
    string "anyURI"
\end{lstlisting}

Note about the \textit{options} attribute: it defines the available values, separated by a coma and surrounded by brackets. It is both used as a type constraint and for parser optimization. The combined default value must be an available option, like the following example.

\lstset{ style=gexf }
\begin{lstlisting}[caption={Options},label=optionsDef]
<graph mode="static">
  <attributes class="node">
    <attribute id="0" title="my-string-attribute" type="string">
        <default>foo</default>
        <options>[foo, bar, foobar]</options>
    </attribute>
    <attribute id="1" title="my-integer-attribute" type="integer">
        <default>5</default>
        <options>[1, 2, 5, 6]</options>
    </attribute>
  </attributes>
  ...
</graph>
\end{lstlisting}

\subsubsection{Defining Attribute Values}

You may understand attributes while looking at this node definition. Besides native fields (id, label), node values are set for three attributes. Omitting an attribute will set the default value as its value. If no default value is set, the value will be empty.

\lstset{ style=gexf }
\begin{lstlisting}[caption={Node Attributes},label=nodeattributes]
<node id="0" label="Hello world">
  <attvalues>
    <attvalue for="0" value="samplevalue"/>
    <attvalue for="1" value="1831"/>
    <attvalue for="2" value="true"/>
  </attvalues>
</node>
\end{lstlisting}

\lstset{ style=gexf }
\begin{lstlisting}[caption={Edge Attributes},label=edgeattributes]
<edge id="0" source="0" target="1">
  <attvalues>
    <attvalue for="0" value="1.5"/>
  </attvalues>
</edge>
\end{lstlisting}

\lstset{ style=rnc }
\begin{lstlisting}[caption={Attribute Values Specification},label=attributeValuesRNC]
attvalues-content =
    element attvalue { attvalue-content }*

attvalue-content =
    attribute for { id-type }
  & attribute value { xsd:string }
\end{lstlisting}

\subsubsection{List Attributes}

GEXF offers list data types liststring, listboolean, listinteger, listlong, listfloat, listdouble, listbyte, listshort, listbigdecimal, listinteger and listchar.

The list format is coma separated surrounded by brackets. For instance \begin{footnotesize}[1, 2, 3]\end{footnotesize} or \begin{footnotesize}[foo, bar]\end{footnotesize}. It supports single and double quotes for text, for instance \begin{footnotesize}['foo', 'bar']\end{footnotesize}. An empty list is simply \begin{footnotesize}[]\end{footnotesize}.

Note that list attributes are an unsafe types! Values are therefore parsed, and this parsing may fail in certain complex cases.

\lstset{ style=gexf }
\begin{lstlisting}[caption={Liststring Definition},label=liststringDef]
<graph mode="static">
  <attributes class="node">
    <attribute id="0" title="my-liststring-attribute" type="liststring">
    </attribute>
  </attributes>
  ...
</graph>
\end{lstlisting}

\lstset{ style=gexf }
\begin{lstlisting}[caption={Liststring usage},label=liststringUse]
<node id="0" label="Hello world">
  <attvalues>
    <attvalue for="0" value="[foo, bar]"/>
  </attvalues>
</node>
\end{lstlisting}

A complete example:

\lstset{ style=gexf }
\begin{lstlisting}[caption={Boolean version},label=boolVersion]
<attributes>
 <attribute id="0" title="hobby" type="liststring">
  </attribute>
</attributes>
<nodes>
 <node id="42" label="a node">
       <attvalues>
           <attvalue for="0" value="[dance, ski]">
       </attvalues>
   </node>
</nodes>
\end{lstlisting}

Also note that when the \textit{options} attribute is used for lists, it gives all possible elements of the list:

\lstset{ style=gexf }
\begin{lstlisting}[caption={Valid values},label=validVal]
<attributes>
 <attribute id="0" title="foo-attr" type="liststring">
   <options>[foo1, foo2, foo3]</options>
 </attribute>
</attributes>
<nodes>
 <node id="42" label="node A">
       <attvalues>
           <attvalue for="0" value="[foo3, foo2]">
       </attvalues>
   </node>
 <node id="43" label="node B">
       <attvalues>
           <attvalue for="0" value="">
       </attvalues>
   </node>
 <node id="44" label="node C">
       <attvalues>
           <attvalue for="0" value="[foo1, foo2, foo3]">
       </attvalues>
   </node>
</nodes>
\end{lstlisting}

\lstset{ style=gexf }
\begin{lstlisting}[caption={Invalid value foo4},label=invalidVal]
...
<node id="42" label="node A">
   <attvalues>
       <attvalue for="0" value="[foo1, foo4]">
   </attvalues>
</node>
\end{lstlisting}

\subsubsection{Omitting attributes}

Since the 1.3 version, GEXF supports a lean definition of attributes directly as part of the \begin{footnotesize}attvalue\end{footnotesize} element. Instead of declaring a set of attributes upfront we rather consider \begin{footnotesize}attvalue\end{footnotesize} as a key/value pair. This is heavier for the parser, produce larger files and may more easily lead to inconsistencies but it can be useful for tools that may not know the list of attributes in advance.

\lstset{ style=gexf }
\begin{lstlisting}[caption={Attribute definition in attvalue },label=attvalueidtype]
...
<node id="42" label="node A">
   <attvalues>
       <attvalue id="visited" type="boolean" value="true">
   </attvalues>
</node>
\end{lstlisting}

A \begin{footnotesize}title\end{footnotesize} may be provided as well but it's optional.

\paragraph{}
The parser would expect consistency in types and titles for a given id. In addition, it isn't recommended to mix this lean and regular format.

\section{Advanced Concepts I: Hierarchy structure} \label{hierarchy}

\subsection{Introduction}

GEXF format allows creating hierarchical graph structure essentially for clustering representation. We modelize both a tree structure of ancestors and descendents, and a flat graph of nodes bound by edges.

\begin{figure}[!ht]
  \begin{center}
  \includegraphics[width=10cm,keepaspectratio=true]{res/hierarchy.png}
  \caption{Graph tree with a virtual edge from a cluster to a leaf}
  \end{center}
\end{figure}

\paragraph{}
Two ways are available:
\begin{enumerate}
 \item Nodes can simply host other nodes and so on.
 \item Each node refers to a parent node id with the XML-attribute \begin{footnotesize}pid\end{footnotesize}.
\end{enumerate}

\lstset{ style=rnc }
\begin{lstlisting}[caption={Hierarchy Specification},label=hierarchyRNC]
# Extension Point
node-content &=
    attribute pid { id-type }?
  & element nodes { nodes-content }?
  & element edges { edges-content }?
\end{lstlisting}

The first style is preferred when the structure written is previously ordered. Sequential reading of this kind of GEXF is safe because no node reference is used. But in the case your program can't provide this, the second way allows writing (and then reading) nodes randomly, but linear reading is at your own risks.

\subsection{Sequential-safe Reading}

\lstset{ style=gexf }
\begin{lstlisting}[caption={First way},label=hierarchy1]
    <graph mode="static" defaultedgetype="directed">
        <nodes>
          <node id="a" label="Kevin Bacon">
            <nodes>
              <node id="b" label="God">
                <nodes>
                  <node id="c" label="human1"/>
                  <node id="d" label="human2"/>
                </nodes>
              </node>
              <node id="e" label="Me">
                <nodes>
                  <node id="f" label="frog1"/>
                  <node id="g" label="frog2"/>
                </nodes>
              </node>
            </nodes>
          </node>
        </nodes>
        <edges>
            <edge id="0" source="b" target="e" />
            <edge id="1" source="c" target="d" />
            <edge id="2" source="g" target="b" />
            <edge id="3" source="f" target="a" />
        </edges>
    </graph>
\end{lstlisting}

Note that edges are not necessarily written at the end:
\lstset{ style=gexf }
\begin{lstlisting}[caption={First way with edges inside clusters},label=hierarchy11]
    <graph mode="static" defaultedgetype="directed">
        <nodes>
          <node id="a" label="Kevin Bacon">
            <nodes>
              <node id="b" label="God">
                <nodes>
                  <node id="c" label="human1"/>
                  <node id="d" label="human2"/>
                </nodes>
                <edges>
                  <edge id="0" source="c" target="d" />
                </edges>
              </node>
              <node id="e" label="Me">
                <nodes>
                  <node id="f" label="frog1"/>
                  <node id="g" label="frog2"/>
                </nodes>
              </node>
            </nodes>
            <edges>
              <edge id="1" source="b" target="e" />
              <edge id="3" source="f" target="a" />
              <edge id="2" source="g" target="b" />
            </edges>
          </node>
        </nodes>
        <edges />
    </graph>
\end{lstlisting}

\subsection{Random Writing}

If you can't structure your graph topology before writing a GEXF file, you may use the second style. Nodes sent to Gephi from a live data source, i.e. a web crawler, are written like this. Note that edges are always written randomly.

\lstset{ style=gexf }
\begin{lstlisting}[caption={Second way},label=hierarchy2]
<nodes>
  <node id="a" label="Kevin Bacon" />
  <node id="b" label="God" pid="a" />
  <node id="c" label="human1" pid="b" />
  <node id="d" label="human2" pid="b" />
  <node id="e" label="Me" pid="a" />
  <node id="f" label="frog1" pid="e" />
  <node id="g" label="frog2" pid="e" />
</nodes>
\end{lstlisting}

With using \begin{footnotesize}pid\end{footnotesize}, node order doesn't matter. An implementation should manage the case when a node reference (pid) is used before the node declaration. This listings could also be:

\lstset{ style=gexf }
\begin{lstlisting}[caption={Second way randomized},label=hierarchy22]
<nodes>
  <node id="g" label="frog2" pid="e" />
  <node id="a" label="Kevin Bacon" />
  <node id="c" label="human1" pid="b" />
  <node id="b" label="God" pid="a" />
  <node id="e" label="Me" pid="a" />
  <node id="d" label="human2" pid="b" />
  <node id="f" label="frog1" pid="e" />
</nodes>
\end{lstlisting}


\section{Advanced Concepts II: Phylogeny structure} \label{phylogeny}

Multiple parents can be addressed with the following syntax, where a and b are c's parents:
\lstset{ style=gexf }
\begin{lstlisting}[caption={Multiple parents},label=phylogeny1]
<nodes>
  <node id="a" label="cheese">
  <node id="b" label="cherry">
  <node id="c" label="cake">
    <parents>
      <parent for="a" />
      <parent for="b" />
    </parents>
  </node>
</nodes>
\end{lstlisting}

\lstset{ style=rnc }
\begin{lstlisting}[caption={Phylogeny Specification},label=phylogenyRNC]
# Extension Point
node-content &=
    element parents { parents-content }?

# New Point
parents-content =
    element parent { parent-content }*

# New Point
parent-content =
    attribute for { id-type }
\end{lstlisting}

\section{Advanced Concepts III: Dynamics} \label{dynamics}

As networks dynamics is a growing topic of research, GEXF format includes extensive time support. Enable it by setting the \begin{footnotesize}mode\end{footnotesize} attribute of the graph to \textit{dynamic}.

\lstset{ style=gexf }
\begin{lstlisting}[caption={Dynamic Enabled!},label=dynamicEnabled]
<graph mode="dynamic">
  ...
</graph>
\end{lstlisting}

Time in GEXF is encoded in two ways, discrete or continuous.

Discrete, it is an \textit{integer} or a \textit{double}. Continuous, it is encoded as an international standard \textit{date} (yyyy-mm-dd) or a \textit{dateTime} defined by the corresponding \href{http://www.w3.org/TR/xmlschema-2/#dateTime}{XSD Datatype}. If omitted, the default type is \textit{double}. Use the the XML-attribute \begin{footnotesize}timeformat\end{footnotesize} of the graph element to explicitly declare the type.

Both network topology and data have a lifetime. The whole graph, each node, each edge and their respective data values may have time limits, beginning with an XML-attribute \begin{footnotesize}start\end{footnotesize} and ending with a \begin{footnotesize}end\end{footnotesize}. Omitting \begin{footnotesize}start\end{footnotesize} set the past of the thing to infinity, so as for \begin{footnotesize}end\end{footnotesize}. The file creator is responsible for the dates consistency.


\subsection{Example}

\lstset{ style=gexf }
\begin{lstlisting}[caption={A (small) Dynamic Web Graph with continuous time},label=dynwebgraph]
<?xml version="1.0" encoding="UTF-8"?>
<gexf xmlns="http://www.gexf.net/1.3"
       xmlns:xsi="http://www.w3.org/2001/XMLSchema-instance"
       xsi:schemaLocation="http://www.gexf.net/1.3
                             http://www.gexf.net/1.3/gexf.xsd"
      version="1.3">
  <meta lastmodifieddate="2009-03-20">
    <creator>Gephi.org</creator>
    <description>A Web network changing over time</description>
  </meta>
  <graph mode="dynamic" defaultedgetype="directed" timeformat="date"
         start="2009-01-01" end="2009-03-20">
    <attributes class="node" mode="static">
      <attribute id="0" title="url" type="string"/>
      <attribute id="1" title="frog" type="boolean">
        <default>true</default>
      </attribute>
    </attributes>
    <attributes class="node" mode="dynamic">
      <attribute id="2" title="indegree" type="float"/>
    </attributes>
    <nodes>
      <node id="0" label="Gephi" start="2009-03-01">
        <attvalues>
          <attvalue for="0" value="http://gephi.org"/>
          <attvalue for="2" value="1"/>
        </attvalues>
      </node>
      <node id="1" label="Webatlas">
        <attvalues>
          <attvalue for="0" value="http://webatlas.fr"/>
          <attvalue for="2" value="1" end="2009-03-01"/>
          <attvalue for="2" value="2" start="2009-03-01" end="2009-03-10"/>
          <attvalue for="2" value="1" start="2009-03-11"/>
        </attvalues>
      </node>
      <node id="2" label="RTGI" end="2009-03-10">
        <attvalues>
          <attvalue for="0" value="http://rtgi.fr"/>
          <attvalue for="2" value="0" end="2009-03-01"/>
          <attvalue for="2" value="1" start="2009-03-01"/>
        </attvalues>
      </node>
      <node id="3" label="BarabasiLab">
        <attvalues>
          <attvalue for="0" value="http://barabasilab.com"/>
          <attvalue for="1" value="false"/>
          <attvalue for="2" value="0" end="2009-03-01"/>
          <attvalue for="2" value="1" start="2009-03-01"/>
        </attvalues>
      </node>
    </nodes>
    <edges>
      <edge id="0" source="0" target="1" start="2009-03-01"/>
      <edge id="1" source="0" target="2"
             start="2009-03-01" end="2009-03-10"/>
      <edge id="2" source="1" target="0" start="2009-03-01"/>
      <edge id="3" source="2" target="1" end="2009-03-10"/>
      <edge id="4" source="0" target="3" start="2009-03-01"/>
    </edges>
  </graph>
</gexf>
\end{lstlisting}

\subsection{Time representation}

GEXF aims to offer equivalent features for both time representations: \textit{timestamp} and \textit{intervals}. Timestamps are unique points in time while intervals can represent a duration and may include infinity as one of its bounds.

\paragraph{}
The two time representations can't however be mixed in GEXF and should be defined and consistent throughout the graph.

\lstset{ style=gexf }
\begin{lstlisting}[caption={Timestamp Time Representation}]
<graph mode="dynamic" timerepresentation="timestamp">
...
</graph>
\end{lstlisting}

The default time representation is \begin{footnotesize}interval\end{footnotesize} to be backward compatible with previous GEXF versions as the timestamp representation is only introduced in 1.3.

\lstset{ style=gexf }
\begin{lstlisting}[caption={Interval Time Representation}]
<graph mode="dynamic" timerepresentation="interval">
...
</graph>
\end{lstlisting}

Note that intervals are always defined with a pair of attributes \textit{start} and \textit{end}. Omitting one of the two translates into an infinity bound. For instance, \begin{footnotesize}node id="0" label="Foo" start="2005"\end{footnotesize} translates into a \begin{footnotesize}[2005, +INF]\end{footnotesize} interval for the node.

\subsection{Time format}

Time in GEXF can be encoded in two ways: via numbers or via dates and times. This is controlled via the \begin{footnotesize}timeformat\end{footnotesize} attribute set on the graph element.

\paragraph{}
As a number, it is an \textit{integer} or a \textit{double}. As a date, it is encoded as an international standard \textit{date} (yyyy-mm-dd) or a \textit{dateTime} defined by the corresponding \href{http://www.w3.org/TR/xmlschema-2/#dateTime}{XSD Datatype}. If omitted, the default type is \textit{double}.

\lstset{ style=gexf }
\begin{lstlisting}[caption={Double format}]
<graph mode="dynamic" timeformat="double">
  <nodes>
    <node id="0" label="Gephi" start="1.0" />
    <node id="1" label="Gexf" start="2.0" end="3.0" />
    ...
  </nodes>
</graph>
\end{lstlisting}

\lstset{ style=gexf }
\begin{lstlisting}[caption={Date format}]
<graph mode="dynamic" timeformat="date">
  <nodes>
    <node id="0" label="Gephi" start="2003-01-01" />
    <node id="1" label="Gexf" start="2003-01-01" end="2004-01-01" />
    ...
  </nodes>
</graph>
\end{lstlisting}

\lstset{ style=gexf }
\begin{lstlisting}[caption={Datetime format (also with timezone)}]
<graph mode="dynamic" timeformat="dateTime">
  <nodes>
    <node id="0" label="Gephi" start="2012-09-12T15:04:01" />
    <!-- With timezone -->
    <node id="1" label="Gexf" start="2012-11-04T11:00:01+03:00"
      end="2012-15-04T11:00:01+03:00" />
    ...
  </nodes>
</graph>
\end{lstlisting}

The timezone can also set globally for the entire graph via the \begin{footnotesize}timezone\end{footnotesize} graph attribute.

\lstset{ style=gexf }
\begin{lstlisting}[caption={Global timezone setting}]
<graph mode="dynamic" timeformat="dateTime" timezone="America/Los_Angeles">
...
</graph>
\end{lstlisting}

\subsection{Dynamic Topology}

Graph nodes and edges can exist at various points in time. Existence in time can be represented in different ways.

\subsubsection{Unique interval or timestamp}

The simplest way is by setting a pair of \begin{footnotesize}start\end{footnotesize} and \begin{footnotesize}end\end{footnotesize} for the interval time representation and \begin{footnotesize}timestamp\end{footnotesize} for the timestamp time representation.

\lstset{ style=gexf }
\begin{lstlisting}[caption={Node Scope Example with Intervals}]
<graph mode="dynamic" timerepresentation="interval">
  <nodes>
    <node id="0" label="Hello" start="2019-01-01" end="2019-02-01" />
    <node id="1" label="World" start="2019-01-15" end="2019-03-20" />
    ...
  </nodes>
</graph>
\end{lstlisting}

\lstset{ style=gexf }
\begin{lstlisting}[caption={Node Scope Example with Timestamps}]
<graph mode="dynamic" timerepresentation="timestamp">
  <nodes>
    <node id="0" label="Hello" timestamp="2019-01-01" />
    <node id="1" label="World" timestamp="2019-03-20" />
    ...
  </nodes>
</graph>
\end{lstlisting}

Each edge must declare time limits inside the join scope of its \begin{footnotesize}source\end{footnotesize} and \begin{footnotesize}target\end{footnotesize} nodes:
\begin{itemize}
 \item edge.start $\le$ (source.start and target.start)
 \item edge.end   $\ge$ (source.end   and target.end)
\end{itemize}

\lstset{ style=gexf }
\begin{lstlisting}[caption={Edge Scope Example}]
<nodes>
  <node id="0" label="Hello" start="2009-01-01" end="2009-02-01" />
  <node id="1" label="World" start="2009-01-15" end="2009-03-20" />
  ...
</nodes>
<edges>
  <edge id="0" source="0" target="1" start="2009-01-20" end="2009-02-01"/>
</edges>
\end{lstlisting}

Important: \begin{footnotesize}start\end{footnotesize} and \begin{footnotesize}end\end{footnotesize} values are inclusive, i.e. the following line is allowed:

\lstset{ style=gexf }
\begin{lstlisting}[caption={Smallest time scope}]
<edge id="0" source="0" target="1" start="2009-01-20" end="2009-01-20"/>
\end{lstlisting}

And of course the \begin{footnotesize}end\end{footnotesize} value must be equal or later than the \begin{footnotesize}start\end{footnotesize} value.

\subsubsection{Spells}

If a node or an edge exists only at some timeranges, we use the concept of spells. Use the xml-element \begin{footnotesize}spells\end{footnotesize} for topology like this:

\lstset{ style=gexf }
\begin{lstlisting}[caption={Node with multiple spells (Intervals)}]
<graph mode="dynamic" timeformat="date" timerepresentation="interval">
  <nodes>
    <node id="0" label="Hello">
      <spells>
        <spell start="2009-01-01" end="2009-01-15" />
        <spell start="2009-01-30" end="2009-02-01" />
      </spells>
    </node>
    ...
  </nodes>
</graph>
\end{lstlisting}

If no \begin{footnotesize}start\end{footnotesize} is provided, the spell begins with the network. If no \begin{footnotesize}end\end{footnotesize} is provided, the spell ends with the network. If two spells are covering a same period of time, parsers should consider them as a unique spell.

\lstset{ style=gexf }
\begin{lstlisting}[caption={Node with multiple spells (Timestamps)}]
<graph mode="dynamic" timeformat="date" timerepresentation="timestamp">
  <nodes>
    <node id="0" label="Hello">
      <spells>
        <spell timestamp="2009-01-15" />
        <spell timestamp="2009-02-01" />
      </spells>
    </node>
    ...
  </nodes>
</graph>
\end{lstlisting}

If spells are provided, only their content are taken into account and other time attributes on the element like \begin{footnotesize}timestamp\end{footnotesize} will be ignored.

Important: Intervals can't overlap for an element.

\lstset{ style=gexf }
\begin{lstlisting}[caption={Overlapping intervals (not allowed)}]
<graph mode="dynamic" timeformat="date" timerepresentation="interval">
  <nodes>
    <node id="0" label="Hello">
      <spells>
        <spell start="2009-01-01" end="2009-01-15" />
        <spell start="2009-01-05" end="2009-02-01" />
      </spells>
    </node>
  ...
  </nodes>
</graph>
\end{lstlisting}

\subsubsection{Alternative to spells}

For a more compact spell representation, GEXF also offers \begin{footnotesize}timestamps\end{footnotesize} and \begin{footnotesize}intervals\end{footnotesize} attributes as lists. It's however less safe as it relies more heavily on the parser. Make sure to check the following examples.

\lstset{ style=gexf }
\begin{lstlisting}[caption={Timestamp list (Date)}]
<graph mode="dynamic" timerepresentation="timestamp" timeformat="date">>
  <nodes>
    <node id="0" label="Hello" timestamps="<[2019-03-20, 2019-03-21]>" />
    ...
  </nodes>
</graph>
\end{lstlisting}

\lstset{ style=gexf }
\begin{lstlisting}[caption={Timestamp list (Double)}]
<graph mode="dynamic" timerepresentation="timestamp" timeformat="double">>
  <nodes>
    <node id="0" label="Hello" timestamps="<[1, 2, 21.0, 124.0]>" />
    ...
  </nodes>
</graph>
\end{lstlisting}

\lstset{ style=gexf }
\begin{lstlisting}[caption={Interval list (Double)}]
<graph mode="dynamic" timerepresentation="timestamp" timeformat="double">>
  <nodes>
    <node id="0" label="Hello" intervals="<[4.0, 14.0]; [21.0, 124.0]>" />
    ...
  </nodes>
</graph>
\end{lstlisting}

Note that the separator between intervals is a semi-colon as the comma is reserved for the interval itself.

\lstset{ style=rnc }
\begin{lstlisting}[caption={Dynamic Topology Specification},label=dyntopoRNC]
# Extension Point
graph-content &=
    attribute timeformat { timeformat-type }?
  & (
      ( attribute start { time-type }?
      | attribute startopen { time-type }?)
      &
      ( attribute end { time-type }?
      & attribute endopen { time-type }?)
  )

# Extension Point
node-content &= (
      ( attribute start { time-type }?
      | attribute startopen { time-type }?)
      &
      ( attribute end { time-type }?
      & attribute endopen { time-type }?)
  )
  & element spells { spells-content }?

# Extension Point
edge-content &= (
      ( attribute start { time-type }?
      | attribute startopen { time-type }?)
      &
      ( attribute end { time-type }?
      & attribute endopen { time-type }?)
  )
  & element spells { spells-content }?
\end{lstlisting}

\paragraph{}
About the weight: dynamic weight can be used with the reserved \textit{title} "weight" in attributes. In dynamic mode, the static XML-attribute \textit{weight} should be ignored if the dynamic one is provided.

\subsection{Dynamic Data}

Node and edges data can take different values over time. Attributes must be declared as dynamic, allowing values to exist for a given time.

\subsubsection{Declaring Dynamic Attributes}

\lstset{ style=gexf }
\begin{lstlisting}[caption={Indegree may change over time}]
<gexf ...>
  ...
  <graph mode="dynamic" defaultedgetype="directed">
    <attributes class="node">
      <attribute id="2" title="indegree" type="float"/>
      ...
    </attributes>
    ...
  </graph>
</gexf>
\end{lstlisting}

\lstset{ style=rnc }
\begin{lstlisting}[caption={Dynamic Attributes Specification},label=dyndataRNC]
# Extension Point
attributes-content &= (
      ( attribute start { time-type }?
      | attribute startopen { time-type }?)
      &
      ( attribute end { time-type }?
      & attribute endopen { time-type }?)
  )
\end{lstlisting}

\subsubsection{Defining Dynamic Values}

Attvalues have their scopes limited by the xml-attributes \begin{footnotesize}start\end{footnotesize} and \begin{footnotesize}end\end{footnotesize}.

\lstset{ style=gexf }
\begin{lstlisting}[caption={Data value changing over time}]
<node id="3" label="BarabasiLab">
  <attvalues>
    <attvalue for="2" value="0" start="2009-01-01" end="2009-03-01"/>
    <attvalue for="2" value="1" start="2009-03-02" end="2009-03-10"/>
  </attvalues>
</node>
\end{lstlisting}

\lstset{ style=gexf }
\begin{lstlisting}[caption={Using timestamp representation}]
<node id="3" label="BarabasiLab">
  <attvalues>
    <attvalue for="2" value="15.0" timestamp="2000" />
    <attvalue for="2" value="20.0" timestamp="2005" />
    <attvalue for="2" value="25.0" timestamp="2010" />
  </attvalues>
</node>
\end{lstlisting}

\lstset{ style=rnc }
\begin{lstlisting}[caption={Dynamic Values Specification},label=dynvalRNC]
# Extension Point
attributes-content &= (
      ( attribute start { time-type }?
      | attribute startopen { time-type }?)
      &
      ( attribute end { time-type }?
      & attribute endopen { time-type }?)
  )
\end{lstlisting}

\subsubsection{Dynamic Values and Spells}

If an \begin{footnotesize}attvalue\end{footnotesize} is covering a period out of any spell, this period should be ignored by parsers. In the following example, the day 2009-01-03 is ignored:

\lstset{ style=gexf }
\begin{lstlisting}[caption={Spells and attvalues}]
<gexf ...>
  ...
  <graph mode="dynamic">
    <node id="0" label="Hello">
      <attvalues>
        <attvalue for="0" value="1" start="2009-01-01" end="2009-01-05"/>
      </attvalues>
      <spells>
        <spell start="2009-01-01" end="2009-01-02" />
        <spell start="2009-01-04" end="2009-01-05" />
      </spells>
    </node>
    ...
  </graph>
</gexf>
\end{lstlisting}

\lstset{ style=rnc }
\begin{lstlisting}[caption={Spells Specification},label=dynspellsRNC]
# New Point
spells-content =
    element spell { spell-content }+

# New Point
spell-content = (
      ( attribute start { time-type }?
      | attribute startopen { time-type }?)
      &
      ( attribute end { time-type }?
      & attribute endopen { time-type }?)
  )
\end{lstlisting}

\section{Advanced Concepts IV: Extending GEXF} \label{extendgexf}

GEXF is designed to be easily extensible. Additional namespaces are defined by an XML Schema. The default namespace is always the gexf namespace. Gephi team actually provides a module for storing visualization data called \textit{viz}.

\subsection{VIZ module} \label{viz}

Using the visualization module must be declared by adding the XML Attribute \begin{footnotesize}xmlns:viz="http://www.gexf.net/1.3/viz"\end{footnotesize} to the document namespaces. The \begin{footnotesize}xsi:schemaLocation\end{footnotesize} attribute includes the XML-Schema declaration of the VIZ module. The RelaxNG Compact specification is available in \href{http://www.gexf.net/1.3/viz.rnc}{viz.rnc}, and independent XSD declaration in \href{http://www.gexf.net/1.3/viz.xsd}{viz.xsd}.

\paragraph{}
Color, position, size and shape are stored as attributes.

\subsubsection{Node Example}

The following gexf contains a node having a color, a position, a shape and a specified size.

\lstset{ style=gexf }
\begin{lstlisting}[caption={VIZ Attributes},label=vizattr]
<gexf xmlns="http://www.gexf.net/1.3"
      xmlns:viz="http://www.gexf.net/1.3/viz">
...
  <node ... >
    <viz:color r="239" g="173" b="66" a="0.5"/>
    <viz:position x="15.783598" y="40.109245" z="0.0"/>
    <viz:size value="2.0375757"/>
    <viz:shape value="disc"/>
  </node>
...
</gexf>
\end{lstlisting}

\subsubsection{Edge Example}

The following gexf contains an edge having a color, a thickness and a shape.

\lstset{ style=gexf }
\begin{lstlisting}[caption={VIZ Attributes},label=vizattr]
<gexf xmlns="http://www.gexf.net/1.3"
      xmlns:viz="http://www.gexf.net/1.3/viz">
...
  <edge ... >
    <viz:color r="157" g="213" b="78"/>
    <viz:thickness value="5.124"/>
    <viz:shape value="solid"/>
  </edge>
...
</gexf>
\end{lstlisting}

\subsubsection{Color}

Colors are defined by the \href{http://en.wikipedia.org/wiki/RGBA}{RGBA color model}. Each XML-attribute value \begin{footnotesize}r\end{footnotesize}, \begin{footnotesize}g\end{footnotesize} or \begin{footnotesize}b\end{footnotesize} is hence an integer from 0 to 255, and the alpha value \begin{footnotesize}a\end{footnotesize} is a float from 0.0 to 1.0.

\lstset{ style=gexf }
\begin{lstlisting}[caption={VIZ Color Declaration},label=vizcolor]
<viz:color r="239" g="173" b="66" a="0.5"/>
\end{lstlisting}

\lstset{ style=rnc }
\begin{lstlisting}[caption={Color Specification},label=colorRNC]
# Extension Point
node-content &=
    element color { color-content }?

# New Point
color-content =
    attribute r { color-channel }
  & attribute g { color-channel }
  & attribute b { color-channel }
  & attribute a { alpha-channel }?

# Datatypes

color-channel =
    xsd:nonNegativeInteger { maxInclusive = "255" }

alpha-channel = [ a:defaultValue = "1.0" ]
    xsd:float { minInclusive = "0.0" maxInclusive = "1.0" }
\end{lstlisting}

Colors can alternatively be defined with a unique \begin{footnotesize}hex\end{footnotesize} attribute. It can still be combined with the alpha parameter

\lstset{ style=gexf }
\begin{lstlisting}[caption={Color Hex Attribute},label=vizhex]
<viz:color hex="#FF7700" alpha="0.5"/>
\end{lstlisting}

\subsubsection{Position}

Space positions are set in three dimensions called \begin{footnotesize}x\end{footnotesize}, \begin{footnotesize}y\end{footnotesize} and \begin{footnotesize}z\end{footnotesize}. Note that Gephi associates \begin{footnotesize}z\end{footnotesize} (optional) as the height, and most of layout algorithms only use \begin{footnotesize}x\end{footnotesize} and \begin{footnotesize}y\end{footnotesize}. They are floats.

\lstset{ style=gexf }
\begin{lstlisting}[caption={VIZ Position Declaration},label=vizposition]
<viz:position x="15.783598" y="40.109245" z="0.0"/>
\end{lstlisting}

\lstset{ style=rnc }
\begin{lstlisting}[caption={Position Specification},label=positionRNC]
# Extension Point
node-content &=
  element position { position-content }?

# New Point
position-content =
    attribute x { space-point }
  & attribute y { space-point }
  & attribute z { space-point }

# Datatype
space-point =
    xsd:float
\end{lstlisting}

\subsubsection{Size}

Node size is a scale. It is set to \textit{1.0} by default and is a non-negative float. Network viz softwares assume that an object representing a node of size \textit{2.0} is twice bigger as one of \textit{1.0}.

\lstset{ style=gexf }
\begin{lstlisting}[caption={VIZ Size Declaration},label=vizsize]
<viz:size value="2.0375757"/>
\end{lstlisting}

\lstset{ style=rnc }
\begin{lstlisting}[caption={Size Specification},label=sizeRNC]
# Extension Point
node-content &=
  element size { size-content }?

# New Point
size-content =
    attribute value { size-type }

# Datatype
size-type = [ a:defaultValue = "1.0" ]
    xsd:float { minInclusive = "0.0"}
\end{lstlisting}

\subsubsection{Thickness}

Edge thickness is a scale. It is set to \textit{1.0} by default and is a non-negative float. Network viz softwares assume that an object representing an edge of thickness \textit{2.0} is twice bigger as one of \textit{1.0}.

\lstset{ style=gexf }
\begin{lstlisting}[caption={VIZ Thickness Declaration},label=vizthickness]
<viz:thickness value="2.0375757"/>
\end{lstlisting}

\lstset{ style=rnc }
\begin{lstlisting}[caption={Thickness Specification},label=thicknessRNC]
# Extension Point
edge-content &=
  element thickness { thickness-content }?

# New Point
thickness-content =
    attribute value { thickness-type }

# Datatype
thickness-type = [ a:defaultValue = "1.0" ]
    xsd:float { minInclusive = "0.0"}
\end{lstlisting}

\subsubsection{Node Shape}

Default node is shaped as a disc. Four shapes are proposed: \textit{disc}, \textit{square}, \textit{triangle} and \textit{diamond}. Images require an additional xml-attribute to set their location: \begin{footnotesize}uri\end{footnotesize}.

\lstset{ style=rnc }
\begin{lstlisting}[caption={Node Shape Specification},label=nshapeRNC]
# Extension Point
node-content &=
  element shape { node-shape-content }?

# New Point
node-shape-content =
    attribute value { node-shape-type }
   & attribute uri { xsd:anyURI }?

# Datatype
node-shape-type =  [ a:defaultValue = "disc" ]
    string "disc" |
    string "square" |
    string "triangle" |
    string "diamond" |
    string "image"
\end{lstlisting}

\lstset{ style=gexf }
\begin{lstlisting}[caption={Image Declaration},label=vizimage]
<viz:shape value="image" uri="http://my.image.us/blah.jpg"/>
\end{lstlisting}

\subsubsection{Edge Shape}

Default edge is shaped as solid. Four shapes are proposed: \textit{solid}, \textit{dotted}, \textit{dashed} and \textit{double}.

\lstset{ style=rnc }
\begin{lstlisting}[caption={Edge Shape Specification},label=eshapeRNC]
# Extension Point
edge-content &=
  element shape { edge-shape-content }?

# New Point
edge-shape-content =
    attribute value { edge-shape-type }

# Datatype
edge-shape-type =  [ a:defaultValue = "solid" ]
    string "solid" |
    string "dotted" |
    string "dashed" |
    string "double"
\end{lstlisting}

\section{Advices: Parser optimization} \label{advices}

This section provides some good tips to write parser-friendly files.
\begin{itemize}
 \item Always place the edges after the nodes, as it is mandatory since version 1.2. Some parsers, depending on their implementation, may reject an edge if its linked nodes haven't been declared before, due to conceptual or data integrity reason.
 \item Use the \textit{count} XML-attribute in \textit{nodes} and \textit{edges} declaration: the parser will know how much memory it should allocate, and will speed up the file reading. Note that count only refers to direct children, not the whole sub-graph!
 \item Prefer \textit{liststring} to \textit{string} attributes if you can. A smart parser will store the strings in one place, and just set pointers to them from the related nodes/edges.
 \item Identifiers may be interpreted as integers if you only use numbers. We encourage this practice, as an integer takes much less size in memory than an equivalent string. Tell the parser to optimize IDs storage by filling the optional graph XML-attribute called \textit{idtype} with \textit{string}, \textit{integer} or \textit{long}.
\end{itemize}

\section{Web Services} \label{ws}

The content-type for serving GEXF files via HTTP/REST is \textit{application/gexf+xml}.

\section{Resources} \label{resources}

Specifications are available on GitHub at \href{https://github.com/gephi/gexf/}{https://github.com/gephi/gexf/}. Contributions and feedback are welcome.

\paragraph{}
The website \href{https://www.gexf.net}{https://www.gexf.net} contains additional examples and links to popular libraries capable of reading and writing the format.

\section{Changelog} \label{changelog}

Note that we used to use "draft" in certain version (e.g "1.2draft") up until 1.3 when we decided to simply use full versions numbers.

\paragraph{}
The GEXF specification history:

\subsection{Version 1.3}

\begin{itemize}
\item Add `kind` attribute on `edge` to support multi-graph (i.e. parallel edges)
\item Add the possibility to omit the upfront `\<attribute`\> definition and rather define the attribute `id` and `type` in the `\<attvalue\>` element
\item The edge `weight` is now a `double` instead of a `float`
\item Add `xsd:long`as possible `idtype` on `\<graph\>`
\item Add new attribute types `bigdecimal`, `biginteger`, `char`, `short` and `byte`
\item Add new list attributes like `listboolean` or `listinteger` for each atomic type
\item Dynamics
\begin{itemize}
\item Add a `timezone` attribute on `\<graph\>` to use as a timezone in case it's omitted in the element timestamps
\item Open intervals attributes `startopen` and `endopen` are removed. Use regular inclusive `start` and `end` instead
\item Remove `mode`, `start` and `end` attributes on `\<attributes\>` as it was redundant with `\<graph\>` attributes
\item Timestamp support
\begin{itemize}
\item Add the ability to represent time with single timestamps instead of intervals. We want feature parity between the two time representations but note they can't be mixed.
\item Add a `timerepresentation` enum in `\<graph\>` with either `interval` (default) or `timestamp` to configure the way the time is represented
\item Add `timestamp` attribute to `\<node\>`, `\<edge\>`, `\<spell\>` and `\<attvalue\>` to support this new time representation
\end{itemize}
\item Alternative to spell elements
\begin{itemize}
\item Add a `timestamps` attribute to `\<node\>` and `\<edge\>` to represent a list of timestamps without having to use spells
\item Similarly, add a `intervals` attribute to `\<node\>` and `\<edge\>`
\end{itemize}
\item New slice mode
\begin{itemize}
\item The optional `mode` attribute on `\<graph\>` now has an additional `slice` value, in addition of `static` and `dynamic`. With slice, the expectation is that the `\<graph\>` also has either a `timestamp` or `start` / `end` intervals.
\item Add a `timestamp` attribute on `\<graph\>` to characterise the slice this graph represent
\item Change the meaning of the `start` and `end` attributes on `\<graph\>` to either characterise the slide instead of the time bounds, which should rather be inferred
\end{itemize}
\end{itemize}
\item Viz
\begin{itemize}
\item Add `hex` attribute on `\<color\>` so it can support values like `#FF00FF`
\item The `z` position is no longer required
\item Dynamic attributes like `start`, `end` or child elements `\<spells\>` are no longer supported for viz attributes. To represent viz attributes over time, an alternative is to create multiple graphs each representing a slice.
\end{itemize}
\end{itemize}

\subsection{Version 1.2}

\begin{itemize}
\item The node and edge `label` attribute are now optional
\item \<meta\> should be placed before \<graph\>
\item Dynamics
\begin{itemize}
\item Rename the `timetype` attribute to `timeformat`. This attribute is set on `\<graph\>` to specify how time information is encoded, either like a date or like a double
\item The `timeformat` is currently either `float` or `date` and default value is `date`. The `float` type is replaced by `double`, and is now the default value
\item Added `timeformat` types `integer` and `dateTime`. DateTime is equivalent to timestamps
\item Add open intervals (non-inclusive): `startopen` and `endopen` attributes
\item `\<slices\>` and `\<slice\>` are renamed `\<spells\>` and `\<spell\>` respectively because slices are a different concept as remarked
\end{itemize}
\item Viz
\begin{itemize}
\item Add viz attributes support for dynamics
\item Add the alpha channel to RGB. Colors are now encoded in RGBA. It is a float from 0.0 (invisible) to 1.0 (fully visible). If omitted, the default alpha-value is 1.0 (no transparency).
\end{itemize}
\end{itemize}

\subsection{Version 1.1}

Modules are stabilized and new ones appear: hierarchy and phylogeny.

\subsection{Version 1.0}

First specification. Basic topology, associated data and dynamics attempt constitute the core, plus a visualization extension.

\printindex

\end{document}
